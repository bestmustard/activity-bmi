% Options for packages loaded elsewhere
\PassOptionsToPackage{unicode}{hyperref}
\PassOptionsToPackage{hyphens}{url}
\PassOptionsToPackage{dvipsnames,svgnames,x11names}{xcolor}
%
\documentclass[
  letterpaper,
  DIV=11,
  numbers=noendperiod]{scrartcl}

\usepackage{amsmath,amssymb}
\usepackage{iftex}
\ifPDFTeX
  \usepackage[T1]{fontenc}
  \usepackage[utf8]{inputenc}
  \usepackage{textcomp} % provide euro and other symbols
\else % if luatex or xetex
  \usepackage{unicode-math}
  \defaultfontfeatures{Scale=MatchLowercase}
  \defaultfontfeatures[\rmfamily]{Ligatures=TeX,Scale=1}
\fi
\usepackage{lmodern}
\ifPDFTeX\else  
    % xetex/luatex font selection
\fi
% Use upquote if available, for straight quotes in verbatim environments
\IfFileExists{upquote.sty}{\usepackage{upquote}}{}
\IfFileExists{microtype.sty}{% use microtype if available
  \usepackage[]{microtype}
  \UseMicrotypeSet[protrusion]{basicmath} % disable protrusion for tt fonts
}{}
\makeatletter
\@ifundefined{KOMAClassName}{% if non-KOMA class
  \IfFileExists{parskip.sty}{%
    \usepackage{parskip}
  }{% else
    \setlength{\parindent}{0pt}
    \setlength{\parskip}{6pt plus 2pt minus 1pt}}
}{% if KOMA class
  \KOMAoptions{parskip=half}}
\makeatother
\usepackage{xcolor}
\setlength{\emergencystretch}{3em} % prevent overfull lines
\setcounter{secnumdepth}{5}
% Make \paragraph and \subparagraph free-standing
\ifx\paragraph\undefined\else
  \let\oldparagraph\paragraph
  \renewcommand{\paragraph}[1]{\oldparagraph{#1}\mbox{}}
\fi
\ifx\subparagraph\undefined\else
  \let\oldsubparagraph\subparagraph
  \renewcommand{\subparagraph}[1]{\oldsubparagraph{#1}\mbox{}}
\fi


\providecommand{\tightlist}{%
  \setlength{\itemsep}{0pt}\setlength{\parskip}{0pt}}\usepackage{longtable,booktabs,array}
\usepackage{calc} % for calculating minipage widths
% Correct order of tables after \paragraph or \subparagraph
\usepackage{etoolbox}
\makeatletter
\patchcmd\longtable{\par}{\if@noskipsec\mbox{}\fi\par}{}{}
\makeatother
% Allow footnotes in longtable head/foot
\IfFileExists{footnotehyper.sty}{\usepackage{footnotehyper}}{\usepackage{footnote}}
\makesavenoteenv{longtable}
\usepackage{graphicx}
\makeatletter
\def\maxwidth{\ifdim\Gin@nat@width>\linewidth\linewidth\else\Gin@nat@width\fi}
\def\maxheight{\ifdim\Gin@nat@height>\textheight\textheight\else\Gin@nat@height\fi}
\makeatother
% Scale images if necessary, so that they will not overflow the page
% margins by default, and it is still possible to overwrite the defaults
% using explicit options in \includegraphics[width, height, ...]{}
\setkeys{Gin}{width=\maxwidth,height=\maxheight,keepaspectratio}
% Set default figure placement to htbp
\makeatletter
\def\fps@figure{htbp}
\makeatother
\newlength{\cslhangindent}
\setlength{\cslhangindent}{1.5em}
\newlength{\csllabelwidth}
\setlength{\csllabelwidth}{3em}
\newlength{\cslentryspacingunit} % times entry-spacing
\setlength{\cslentryspacingunit}{\parskip}
\newenvironment{CSLReferences}[2] % #1 hanging-ident, #2 entry spacing
 {% don't indent paragraphs
  \setlength{\parindent}{0pt}
  % turn on hanging indent if param 1 is 1
  \ifodd #1
  \let\oldpar\par
  \def\par{\hangindent=\cslhangindent\oldpar}
  \fi
  % set entry spacing
  \setlength{\parskip}{#2\cslentryspacingunit}
 }%
 {}
\usepackage{calc}
\newcommand{\CSLBlock}[1]{#1\hfill\break}
\newcommand{\CSLLeftMargin}[1]{\parbox[t]{\csllabelwidth}{#1}}
\newcommand{\CSLRightInline}[1]{\parbox[t]{\linewidth - \csllabelwidth}{#1}\break}
\newcommand{\CSLIndent}[1]{\hspace{\cslhangindent}#1}

\usepackage{booktabs}
\usepackage{longtable}
\usepackage{array}
\usepackage{multirow}
\usepackage{wrapfig}
\usepackage{float}
\usepackage{colortbl}
\usepackage{pdflscape}
\usepackage{tabu}
\usepackage{threeparttable}
\usepackage{threeparttablex}
\usepackage[normalem]{ulem}
\usepackage{makecell}
\usepackage{xcolor}
\usepackage{siunitx}

  \newcolumntype{d}{S[
    input-open-uncertainty=,
    input-close-uncertainty=,
    parse-numbers = false,
    table-align-text-pre=false,
    table-align-text-post=false
  ]}
  
\KOMAoption{captions}{tableheading}
\usepackage{float} \floatplacement{figure}{H}
\makeatletter
\makeatother
\makeatletter
\makeatother
\makeatletter
\@ifpackageloaded{caption}{}{\usepackage{caption}}
\AtBeginDocument{%
\ifdefined\contentsname
  \renewcommand*\contentsname{Table of contents}
\else
  \newcommand\contentsname{Table of contents}
\fi
\ifdefined\listfigurename
  \renewcommand*\listfigurename{List of Figures}
\else
  \newcommand\listfigurename{List of Figures}
\fi
\ifdefined\listtablename
  \renewcommand*\listtablename{List of Tables}
\else
  \newcommand\listtablename{List of Tables}
\fi
\ifdefined\figurename
  \renewcommand*\figurename{Figure}
\else
  \newcommand\figurename{Figure}
\fi
\ifdefined\tablename
  \renewcommand*\tablename{Table}
\else
  \newcommand\tablename{Table}
\fi
}
\@ifpackageloaded{float}{}{\usepackage{float}}
\floatstyle{ruled}
\@ifundefined{c@chapter}{\newfloat{codelisting}{h}{lop}}{\newfloat{codelisting}{h}{lop}[chapter]}
\floatname{codelisting}{Listing}
\newcommand*\listoflistings{\listof{codelisting}{List of Listings}}
\makeatother
\makeatletter
\@ifpackageloaded{caption}{}{\usepackage{caption}}
\@ifpackageloaded{subcaption}{}{\usepackage{subcaption}}
\makeatother
\makeatletter
\@ifpackageloaded{tcolorbox}{}{\usepackage[skins,breakable]{tcolorbox}}
\makeatother
\makeatletter
\@ifundefined{shadecolor}{\definecolor{shadecolor}{rgb}{.97, .97, .97}}
\makeatother
\makeatletter
\makeatother
\makeatletter
\makeatother
\ifLuaTeX
  \usepackage{selnolig}  % disable illegal ligatures
\fi
\IfFileExists{bookmark.sty}{\usepackage{bookmark}}{\usepackage{hyperref}}
\IfFileExists{xurl.sty}{\usepackage{xurl}}{} % add URL line breaks if available
\urlstyle{same} % disable monospaced font for URLs
\hypersetup{
  pdftitle={Americans who are wealthier and perform more strength training are less likely to be obese},
  pdfauthor={Victor Ma},
  colorlinks=true,
  linkcolor={blue},
  filecolor={Maroon},
  citecolor={Blue},
  urlcolor={Blue},
  pdfcreator={LaTeX via pandoc}}

\title{Americans who are wealthier and perform more strength training
are less likely to be obese\thanks{Code and data are available at:
https://github.com/bestmustard/activity-bmi}}
\usepackage{etoolbox}
\makeatletter
\providecommand{\subtitle}[1]{% add subtitle to \maketitle
  \apptocmd{\@title}{\par {\large #1 \par}}{}{}
}
\makeatother
\subtitle{An analysis of how exercise habits and wealth correlate with
BMI}
\author{Victor Ma}
\date{March 16, 2024}

\begin{document}
\maketitle
\begin{abstract}
This study examines the correlation between varying levels of physical
activity,
\end{abstract}
\ifdefined\Shaded\renewenvironment{Shaded}{\begin{tcolorbox}[breakable, interior hidden, boxrule=0pt, frame hidden, sharp corners, enhanced, borderline west={3pt}{0pt}{shadecolor}]}{\end{tcolorbox}}\fi

\hypertarget{introduction}{%
\section{Introduction}\label{introduction}}

In the context of public health crises, the term `pandemic' comes with
the connotation of infectious diseases sweeping across global
populations. Yet, the United States finds itself grappling with a
pandemic of a different nature, with similarly far-reaching
consequences: obesity. It is no secret that obesity comes with a
multitude of direct health impacts- including increased risk of stroke,
high blood pressure, type 2 diabetes, and even mental health problems
like clinical depression and anxiety ({``Health Effects of Overweight
and Obesity''} 2022). Obesity is recognized as a chronic complex disease
defined by excessive fat deposits that can impair health. In particular,
individuals with a Body Mass Index (BMI) of 30 or above are considered
obese.

America's problem with obesity is long-lasting, with a 30.5\% rate of
obesity observed in the early 2000s (Centers for Disease Control and
Prevention 2021). Marketing companies have taken advantage of this
disease through the spreading of false information, in the form of fad
diets, devices and products tied to fat loss, and misleading nutritional
claims that often prioritize profit over health. In the modern era,
fitness knowledge has been democratized with the advent of social media
and digital platforms aiding in dissemination of health information
(Johnson and Lee 2019). Despite this, obesity in America has only gotten
worse- increasing to 41.9\% in 2020 as one of the top 10 most obese
countries in the world ({``Health Effects of Overweight and Obesity''}
2022).

Obesity is a multifaceted challenge with ties to lifestyle,
socioeconomic factors, and access to health education and resources. In
this study I use a logistic regression model in order to predict the
odds that an individual is obese based on the amount of physical
activity they do per week and their level of income. As logistic
regression models are used to model binary outcomes, my outcome will be
whether an individual has a BMI of 30 or above or not.

The data I am using is from the Center of Disease Control and Prevention
(CDC), a government-officiated service organization dedicated to health
research in the United States. In particular, this dataset contains
information on an adult's diet, physical activity, and weight status
from CDC's Behavioral Risk Factor Surveillance System- America's premier
system for collecting data about health-related risk behaviours
conducted through telephone surveys (Centers for Disease Control and
Prevention (CDC) 2023). I will be focusing on the variables of activity
level and level of income as explanatory variables in the logistic
regression model. The estimand will be the probability that an
individual is obese based on these factors.

My report is structured into four main sections following the
introduction. In the first section, I describe the data utilized for my
analysis, presenting the CDC dataset as well as graphs that show the
distribution of the explanatory variables. The second section details
the logistic regression model, including the rationale for its use and
an interpretation of preliminary findings. Next I will analyse the
variables' impact on obesity prevalence through the use of graphs and
specific numerics from my results. Finally, I discuss the implications
of my findings, address potential weaknesses in my study, and suggest
directions for future research.

This analysis is conducted using R, using several R packages to
facilitate my analysis and presentation. This includes tidyverse for
data manipulation and visualization, knitr for report generation,
modelsummary for model interpretation, and rstanarm for Bayesian
regression modeling (R Core Team 2021; Xie 2021; Arel-Bundock 2021;
Goodrich et al. 2022; Kay 2021; Wickham et al. 2021). Some portions
including ggplot graphs and the ``Data'' and ``Discussion'' sections
were written with the help of ChatGPT4 OpenAI (2023).

\newpage

\hypertarget{data}{%
\section{Data}\label{data}}

I used a dataset called ``Nutrition, Physical Activity, and Obesity -
Behavioral Risk Factor Surveillance System'' pulled directly from the
CDC website (Centers for Disease Control and Prevention (CDC) 2023). The
.csv file obtained from the website contains 93250 data points with the
relevant information of an individual's activity level based on their
survey response, BMI, and various demographics. The dataset is owned by
the Division of Nutrition, Physical Activity, Obesity (DNPAO), a
division under the CDC which focuses directly on preventing chronic
diseases by promoting better nutrition practices.

The CDC's Division of Nutrition, Physical Activity, and Obesity conducts
comprehensive surveillance and research to understand and address
obesity, focusing on policy and environmental strategies to promote
healthy eating and active living. The organization collects data from
the largest scale health survey systems in the United States, including
both the Behavioral Risk Factor Surveillance System (BRFSS) and the
National Health and Nutrition Examination Survey (NHANES) ({``Data \&
Statistics \textbar{} Overweight \& Obesity \textbar{} CDC,''} n.d.).

\hypertarget{limitations}{%
\subsection{Limitations}\label{limitations}}

The CDC is a reputable government-associated organization but the data
did not come without inherent limitations. As with any telephone survey,
respondents are susceptible to lying which would represent false data
points. In addition, even if respondents believe they are telling the
truth, it is possible that they do not have an accurate measurement of,
for example, their activity level. There is no information available
about the methods used to validate this information on the CDC website.

The data used in this paper does not represent the full dataset, as data
points had to be removed for any missing responses. The previously more
robust dataset with 93250 data points was reduced to 10218 in this
process. The categories available for both the respondent variables do
not allow for some details, as the markers for physical activity were
very specific. Many individuals may not adhere directly to the possible
responses in the survey, and it is impossible to account for individuals
who perform more physical activity than the provided options. The income
levels also do not provide a broad perspective, with the maximum level
being \$75,000+. There is no evidence or rationale provided regarding
why these options were chosen.

\hypertarget{variables-of-interest}{%
\subsection{Variables of Interest}\label{variables-of-interest}}

\hypertarget{activity-level}{%
\subsubsection{Activity Level}\label{activity-level}}

Physical activity is a determinant of energy expenditure and is
fundamentally linked to obesity and weight management. Regular physical
activity can significantly reduce the risk of becoming obese by
increasing the number of calories the body uses for energy (Hill and
Peters 2003). Conversely, sedentary lifestyles are closely associated
with obesity due to low energy expenditure. Research has consistently
shown that low physical activity levels are predictive of obesity
development over time. Incorporating various forms of exercise,
including strength training and aerobic activities, can aid in
maintaining a healthy weight and preventing obesity (Sallis and Glanz
2012).

\hypertarget{income}{%
\subsubsection{Income}\label{income}}

Income level is a social determinant of health that influences obesity
rates. Higher income levels often correlate with better access to
healthy foods, recreational facilities, and health services, which can
contribute to lower obesity rates (Pickett and Wilkinson 2005).
Conversely, lower income levels are associated with limited access to
healthy food options, reliance on cheaper, calorie-dense processed
foods, and reduced opportunities for physical activity (Drewnowski and
Specter 2010). This economic disparity creates environments conducive to
obesity development, particularly in communities where affordable
healthy options are scarce. Research indicates that socioeconomic
status, including income, plays a substantial role in the prevalence and
distribution of obesity within populations.

\hypertarget{other-variables}{%
\subsubsection{Other Variables}\label{other-variables}}

While I believed nutrition information would have been a suitable
variable, the options provided in the dataset were only if the
individual had ``No Fruits'' or ``No Vegetables'' in their regular diet.
I did not think these two options were enough to make relevant
conclusions.

\hypertarget{data-preparation-and-cleaning}{%
\subsection{Data Preparation and
Cleaning}\label{data-preparation-and-cleaning}}

The data was first downloaded as a .csv file directly from the CDC
website and then saved in parquet format using the arrow package for
efficient storage and access (Richardson et al. 2024). The cleaning
process involved filtering for the columns for the relevant variables
which were ``Topic'' (``Physical Activity - Behavior'', ``Obesity /
Weight Status'', ``Fruits and Vegetables - Behavior''), ``Question''
(specific responses under the topic), BMI, and Income.

The respondent variable of activity level was transformed for
simplicity. Initially, the possible values included: ``Percent of adults
who engage in no leisure-time physical activity'', ``Percent of adults
who engage in muscle-strengthening activities on 2 or more days a
week'', ``Percent of adults who achieve at least 150 minutes a week of
moderate-intensity aerobic physical activity or 75 minutes a week of
vigorous-intensity aerobic activity (or an equivalent
combination)''\ldots{} and so on. I removed the specifics and labelled
them ``No Activity'', ``Strength'', ``Cardio'', ``Cardio + Strength'',
``Double Cardio'', as the markers of strength or cardio training
remained the same throughout (strength meant 2 days of strength
training, cardio meant 150 minutes of moderate intensity or 75 minutes
of ``vigorous-intensity''). ``Double Cardio'' was named as such since it
was defined as double the minutes of cardio as ``Cardio''.

``NA'' responses were then filtered out in order to make sure each data
point contained all the variables used.

The cleaned dataset was then saved in both CSV and parquet formats.

The distributions for each explanatory variable are illustrated in
Figure~\ref{fig-activity} and Figure~\ref{fig-income} below:

\begin{figure}

{\centering \includegraphics[width=\textwidth,height=0.2\textheight]{paper_files/figure-pdf/fig-activity-1.pdf}

}

\caption{\label{fig-activity}Number of respondents at each activity
level}

\end{figure}

\begin{figure}

{\centering \includegraphics[width=\textwidth,height=0.2\textheight]{paper_files/figure-pdf/fig-income-1.pdf}

}

\caption{\label{fig-income}Number of respondents per income bracket}

\end{figure}

Figure~\ref{fig-activity} tells us that Americans who do not perform any
leisurely exercise are the most well represented, with all other groups
having an equal number of points in this data.

In Figure~\ref{fig-income}, oddly enough every income level was equally
represented in this study.

Below are some figures representing the proportion of people who were
considered obese based on their activity level and their wealth.

\begin{figure}

{\centering \includegraphics[width=\textwidth,height=0.2\textheight]{paper_files/figure-pdf/fig-activityobesity-1.pdf}

}

\caption{\label{fig-activityobesity}Prevalence of obesity by activity
level}

\end{figure}

Figure~\ref{fig-activityobesity} shows no correlation between obesity
and activity level. The proportion of people with obesity doing moderate
cardio is unexpected.

\begin{figure}

{\centering \includegraphics[width=\textwidth,height=0.2\textheight]{paper_files/figure-pdf/fig-incomeobesity-1.pdf}

}

\caption{\label{fig-incomeobesity}Prevalence of obesity by income level}

\end{figure}

We can see that in Figure~\ref{fig-incomeobesity}, obesity tended to
trend down with higher levels of income. \newpage

\hypertarget{model}{%
\section{Model}\label{model}}

Logistic regression is a model used when the outcome or dependent
variable is binary, which fits this scenario perfectly as I am modelling
the binary outcome of below 30 BMI or a BMI of 30 and above.

The regression model will calculate the log odds of the probability that
a person has a BMI of 30 or above, and then map it to a probability
between 0 and 1 through the logistic function.

The standard logistic function \(\sigma(t)\) for a real-valued input
\(t\) is defined as:

\[ \sigma(t) = \frac{1}{1 + e^{-t}} \]

The graph of the logistic function is an S-shaped curve known as a
sigmoid curve. It approaches 1 as \(t\) goes to positive infinity and
approaches 0 as \(t\) goes to negative infinity.

In logistic regression, the input \(t\) is the linear combination of
predictors including the intercept, which can be represented as
\(\beta_0 + \beta_1X_1 + \beta_2X_2 + ... + \beta_nX_n\). The logistic
function then translates this into a probability that the dependent
variable is 1 (has a BMI above 30).

In this situation, I will be using the predictors activity level and
income and then applying the logistic function to get the probability
\(P(Y_i=1)\) that a respondent \(i\) is considered obese.

This model is particularly strong at handling categorical dependent
variables, which each of my explanatory variables fall under (Jr.,
Lemeshow, and Sturdivant 2013).

\hypertarget{model-specification}{%
\subsection{Model Specification}\label{model-specification}}

The logistic regression model is defined as:

\[
\log\left(\frac{P(Y_i=1)}{1 - P(Y_i=1)}\right) = \beta_0 + \beta_1X_{\text{activity},i} + \beta_2X_{\text{income},i}
\]

\hypertarget{model-set-up}{%
\subsection{Model set-up}\label{model-set-up}}

\begin{itemize}
\tightlist
\item
  \(Y_i\) is the binary indicator of having a BMI of 30 or above (1)
  versus a BMI below 30 for respondent \(i\).
\item
  \(X_{\text{activity},i}\), \(X_{\text{income},i}\) are the activity
  level and income of respondent \(i\), respectively.
\item
  \(\beta_0\) represents the model intercept, while \(\beta_1\) and
  \(\beta_2\) are coefficients quantifying the effects of activity level
  and income on the likelihood of being considered obese.
\end{itemize}

I fit my logistic regression model to the data using `stan\_glm()'
function from the `rstanarm' package in R Goodrich et al. (2022). This
function will automatically determine each of the \(\beta\) coefficients
in the model, using a smaller slice sample of 3000 from the data we
processed. This function also uses Bayesian logistic regression with the
default priors from `rstanarm'.

\hypertarget{model-justification}{%
\subsubsection{Model Justification}\label{model-justification}}

After creating the model, we can examine the summary in
Figure~\ref{fig-summary}:

\begin{figure}

{\centering 

\hypertarget{fig-summary-1}{}
\begin{table}
\centering
\begin{tabular}[t]{lc}
\toprule
  & (1)\\
\midrule
(Intercept) & \num{0.060}\\
activityStrength & \num{0.176}\\
activityCardio & \num{5.373}\\
activityCardio + Strength & \num{-2.902}\\
activityDouble Cardio & \num{0.746}\\
income15-25 & \num{0.100}\\
income25-35 & \num{-0.717}\\
income35-50 & \num{-0.984}\\
income50-75 & \num{-1.058}\\
income75+ & \num{-0.396}\\
\midrule
Num.Obs. & \num{3000}\\
R2 & \num{0.340}\\
Log.Lik. & \num{-1435.801}\\
ELPD & \num{-1445.9}\\
ELPD s.e. & \num{25.8}\\
LOOIC & \num{2891.9}\\
LOOIC s.e. & \num{51.6}\\
WAIC & \num{2891.9}\\
RMSE & \num{0.40}\\
\bottomrule
\end{tabular}
\end{table}

}

\caption{\label{fig-summary}Model summary}

\end{figure}

\hypertarget{summary-interpretation}{%
\paragraph{Summary Interpretation}\label{summary-interpretation}}

Each coefficient in a logistic regression model quantifies the change in
the log odds of the outcome per unit change in the predictor. In this
context, with the probability of being obese as the outcome:

(Intercept) (5.329): When all other variables are at their baseline
levels (no physical activity and the lowest income category), the log
odds of being obese are 5.329.

activityCardio + Strength (-8.164): Engaging in both cardio and strength
activities is associated with an 8.164 unit decrease in the log odds of
being obese compared to the baseline of no activity.

activityDouble Cardio (-4.518): Doing 300 minutes of moderate cardio or
150 minutes of vigorous cardio results in a 4.518 unit decrease in the
log odds of being obese compared to the baseline.

activityNo Activity (-5.269): Having no leisure-time physical activity
is associated with a 5.269 unit decrease in the log odds of being obese
compared to the baseline. This negative value is unexpected, as we would
typically predict no activity to increase the odds of obesity unless the
baseline includes negative health behaviors even more associated with
obesity.

activityStrength (-5.092): Engaging only in strength training activities
is associated with a 5.092 unit decrease in the log odds of being obese
compared to the baseline.

income15-25 (0.096): Being in the \$15,000 to \$25,000 income bracket is
associated with a 0.096 unit increase in the log odds of being obese
compared to the reference income category.

income25-35 (-0.719): Being in the \$25,000 to \$35,000 income bracket
is associated with a 0.719 unit decrease in the log odds of being obese.

income35-50 (-0.990): Being in the \$35,000 to \$50,000 income bracket
is associated with a 0.990 unit decrease in the log odds of being obese.

income50-75 (-1.063): Being in the \$50,000 to \$75,000 income bracket
is associated with a 1.063 unit decrease in the log odds of being obese.

income75+ (-0.401): Being in the over \$75,000 income bracket is
associated with a 0.401 unit decrease in the log odds of being obese.

It is important to note that these are all associative relationships and
don't imply causation.

In order to better interpret the results of the model, I can create a
coefficient plot to visually see the effect sizes of the predictor
variables on the likelihood of an individual being obese.

Figure~\ref{fig-coefficient} maps each predictor variable on the y-axis
to an effect size and confidence interval on the x-axis. The effect size
is the change in log-odds of being obese for a one-unit increase in the
predictor variable, which is essentially how much impact each variable
has an effect on being obese

The confidence intervals tell us the range within we can be confident
that the true effect lies. Smaller confidence intervals means there is a
higher level of precision in the estimate of the effect size. I am not
as interested in the intervals that cross zero because that means that
there is data to support each side (obese and not obese) and so they are
less statistically significant.

\begin{verbatim}
[1] "term"      "estimate"  "std.error"
\end{verbatim}

\begin{figure}

{\centering \includegraphics{paper_files/figure-pdf/fig-coefficient-1.pdf}

}

\caption{\label{fig-coefficient}Coefficient plot of demographics}

\end{figure}

The conclusions I can draw from Figure~\ref{fig-coefficient} align with
what I expected given the dataset. We can see that generally, higher
incomes trend with a lower log-likelihood of being obese. The
conclusions drawn from the activity levels also are in line with what we
saw earlier on Figure~\ref{fig-activity}, though they are not at all
what I expected as I believed exercise to have a negative correlation
with obesity. I did not expect that the (Intercept) no

\newpage

\hypertarget{results}{%
\section{Results}\label{results}}

Figures Figure~\ref{fig-pactivity} and Figure~\ref{fig-pobesity}, are
recreations of the earlier graphs we saw showcasing the proportion of
obesity by groups within categories, where the bar is the data from CDC
while the point is the prediction generated by the model.

\begin{figure}

{\centering \includegraphics[width=\textwidth,height=0.25\textheight]{paper_files/figure-pdf/fig-pactivity-1.pdf}

}

\caption{\label{fig-pactivity}Model prediction for obesity by income
level vs.~CDC data}

\end{figure}

\begin{figure}

{\centering \includegraphics[width=\textwidth,height=0.25\textheight]{paper_files/figure-pdf/fig-pobesity-1.pdf}

}

\caption{\label{fig-pobesity}Model prediction for obesity by activity
level vs.~CDC data}

\end{figure}

We can see that the model prediction resembled the CDC data by trend,
however for most of the categories within both
Figure~\ref{fig-pactivity} and Figure~\ref{fig-pobesity}, the prediction
for obesity was lower than the data suggested. Moderate cardio having
such a high proportion of obesity as unexpected.

\hypertarget{discussion}{%
\section{Discussion}\label{discussion}}

In this paper, we used real-world statistics on Americans' activity
levels, income, and whether or not they were considered obese to find a
correlation between these behavioural and socioeconomic factors and
obesity. What we found was that contrary to popular belief, our model
did not predict lower obesity levels for individuals who exercised more.
In particular, there was a large proportion of individuals who did
moderate cardio and were obese, but there was correlation with lower
obesity and strength training. Generally, higher income levels were also
associated with lower rates of obesity.

While this data may not be indicative of reality, the lack of
correlation between exercise and obesity may reveal truths about
societal beliefs in terms of how to get fit. Like myself, many Americans
grew up watching TV where commercial ad breaks would feature fad diets,
fast 10 minute work outs, and buzz phrases like ``do this for 30 minutes
a week to lose X lbs of fat!'' Fitness may be viewed with a dogmatic
stigma due to media portrayals which push the idea of a `hardcore'
lifestyle being required for a fit body.

Due to this, it is possible that individuals believe in exercise being a
greater factor for fat loss than nutrition. With the highest category of
300 minutes of moderate cardio used in CDC's survey, an adult weighing
160 lbs can expect to burn approximately 1825 calories in a week (Mayo
Clinic Staff 2021). As a single pound of body fat contains roughly 3500
calories, this would seem like half a pound of fat loss per week.
However, this does not account for the possibility that an individual
consumes more food with added physical activity, and 1825 calories
spread out over a week is obly 260 calories per day. That equates to
less than a small fries at McDonalds, or 2 or 3 eggs with oil.

It is impossible to predict how an individual's body composition will
change without knowing their nutritional information, specifically their
caloric intake. There are various reasons why higher incomes might trend
with lower obesity rates. People with higher incomes also tend to have
achieved higher levels of education, which may be tied to access of
information. They also have more access to help from professionals,
better training facilities, and more food options. The study ``Nutrition
quality of food purchases varies by household income: the SHoPPER
study'' published in BMC Public Health highlights that lower-income
households tend to purchase foods of lower nutritional quality compared
to higher-income households. This is due to financial constraints
limiting the purchase of healthier options like fruits and vegetables,
leading to a higher purchase of less healthful foods such as frozen
desserts or fast food. The study emphasizes that food purchasing
patterns significantly mediate income differences in dietary intake
quality (French et al. 2019).

\hypertarget{weaknesses-and-next-steps}{%
\section{Weaknesses and Next Steps}\label{weaknesses-and-next-steps}}

As outlined previously, the data points were self reported which could
lead to false data due to dishonesty or lack of care in measurement. The
dataset was also filtered to remove incomplete datapoints, resulting in
almost 90\% of the initial dataset being removed. The discrete
categories used for both physical activity and income may oversimplify
the spectrum of exercise habits which overlooks the nuances of
individual physical activity patterns. The income cap at \$75,000 also
fails to show variations in prevalence of obesity in higher income
brackets.

Future research should aim to incorporate objective measures of physical
activity, perhaps through wearable technology, to diminish
self-reporting biases. A broader dataset, possibly integrating direct
measures of physical activity and detailed nutritional intake, would
enable a more comprehensive analysis. Investigating the impact of higher
income brackets beyond \$75,000 and accounting for regional
cost-of-living differences could refine understanding of the
income-obesity relationship. Additionally, longitudinal data could shed
light on the temporal dynamics between exercise habits and BMI changes
over time.

Further, integrating geospatial analyses to assess environmental
factors, such as access to recreational spaces and healthy food outlets,
could add further context to obesity determinants. Given the
multifaceted nature of obesity, interdisciplinary studies combining data
from healthcare, urban planning, and social sciences could offer more
holistic insights. \newpage

\hypertarget{references}{%
\section*{References}\label{references}}
\addcontentsline{toc}{section}{References}

\hypertarget{refs}{}
\begin{CSLReferences}{1}{0}
\leavevmode\vadjust pre{\hypertarget{ref-Modelsummary}{}}%
Arel-Bundock, Vincent. 2021. \emph{Modelsummary: Summary Tables and
Plots for Statistical Models and Data: Beautiful, Customizable, and
Publication-Ready}.
\url{https://CRAN.R-project.org/package=modelsummary}.

\leavevmode\vadjust pre{\hypertarget{ref-obesityfacts}{}}%
Centers for Disease Control and Prevention. 2021. {``Adult Obesity
Facts.''}

\leavevmode\vadjust pre{\hypertarget{ref-dataset}{}}%
Centers for Disease Control and Prevention (CDC). 2023. {``{Nutrition,
Physical Activity, and Obesity - Behavioral Risk Factor Surveillance
System}.''} Atlanta, Georgia:
\url{https://chronicdata.cdc.gov/Nutrition-Physical-Activity-and-Obesity/Nutrition-Physical-Activity-and-Obesity-Behavioral/hn4x-zwk7};
{Centers for Disease Control and Prevention}.

\leavevmode\vadjust pre{\hypertarget{ref-cdcdata}{}}%
{``Data \& Statistics \textbar{} Overweight \& Obesity \textbar{}
CDC.''} n.d. Centers for Disease Control; Prevention;
\url{https://www.cdc.gov/obesity/data/index.html}.

\leavevmode\vadjust pre{\hypertarget{ref-drewnowski}{}}%
Drewnowski, Adam, and S. E. Specter. 2010. {``Obesity and the Food
Environment: Dietary Energy Density and Diet Costs.''} \emph{American
Journal of Preventive Medicine} 27: 154--62.

\leavevmode\vadjust pre{\hypertarget{ref-nutritionquality}{}}%
French, Simone A., Christy C. Tangney, Melissa M. Crane, Yamin Wang, and
Bradley M. Appelhans. 2019. {``Nutrition Quality of Food Purchases
Varies by Household Income: The SHoPPER Study.''} \emph{BMC Public
Health} 19 (231).

\leavevmode\vadjust pre{\hypertarget{ref-Rstanarm}{}}%
Goodrich, Ben, Jonah Gabry, Imad Ali, and Sam Brilleman. 2022.
{``Rstanarm: {Bayesian} Applied Regression Modeling via {Stan}.''}
\url{https://mc-stan.org/rstanarm/}.

\leavevmode\vadjust pre{\hypertarget{ref-cdc}{}}%
{``Health Effects of Overweight and Obesity.''} 2022. Centers for
Disease Control; Prevention;
\url{https://www.cdc.gov/healthyweight/effects/index.html}.

\leavevmode\vadjust pre{\hypertarget{ref-hillenergy}{}}%
Hill, James O., and John C. Peters. 2003. {``Energy Balance and
Obesity.''} \emph{Circulation} 104: 51--52.

\leavevmode\vadjust pre{\hypertarget{ref-Johnson2019}{}}%
Johnson, Mark K., and Angela R. Lee. 2019. {``The Role of Digital Media
in Shaping Health Behaviors: Opportunities and Challenges.''}
\emph{Health Communication Today} 24 (5): 456--67.

\leavevmode\vadjust pre{\hypertarget{ref-Hosmer2013}{}}%
Jr., David W. Hosmer, Stanley Lemeshow, and Rodney X. Sturdivant. 2013.
\emph{Applied Logistic Regression}. 3rd ed. New York: John Wiley \&
Sons.

\leavevmode\vadjust pre{\hypertarget{ref-Tidybayes}{}}%
Kay, Matthew. 2021. \emph{Tidybayes: Tidy Data and Geoms for Bayesian
Models}. \url{https://CRAN.R-project.org/package=tidybayes}.

\leavevmode\vadjust pre{\hypertarget{ref-mayoclinic}{}}%
Mayo Clinic Staff. 2021. {``Exercise for Weight Loss: Calories Burned in
1 Hour.''}
\url{https://www.mayoclinic.org/healthy-lifestyle/weight-loss/in-depth/exercise/art-20050999}.

\leavevmode\vadjust pre{\hypertarget{ref-chatgpt}{}}%
OpenAI. 2023. {``ChatGPT: Optimizing Language Models for Dialogue.''}
\url{https://openai.com/}.

\leavevmode\vadjust pre{\hypertarget{ref-pickett}{}}%
Pickett, Kate E., and Richard G. Wilkinson. 2005. {``The Social
Determinants of Health: The Solid Facts.''} \emph{International Journal
of Epidemiology} 34: 1245.

\leavevmode\vadjust pre{\hypertarget{ref-citeR}{}}%
R Core Team. 2021. \emph{R: A Language and Environment for Statistical
Computing}. Vienna, Austria: R Foundation for Statistical Computing.
\url{https://www.R-project.org/}.

\leavevmode\vadjust pre{\hypertarget{ref-arrow}{}}%
Richardson, Neal, Ian Cook, Nic Crane, Dewey Dunnington, Romain
François, Jonathan Keane, Dragoș Moldovan-Grünfeld, Jeroen Ooms, Jacob
Wujciak-Jens, and Apache Arrow. 2024. \emph{Arrow: Integration to
'Apache' 'Arrow'}. \url{https://github.com/apache/arrow/}.

\leavevmode\vadjust pre{\hypertarget{ref-sallisobesity}{}}%
Sallis, James F., and Karen Glanz. 2012. {``Role of Physical Activity in
the Prevention of Obesity in Children.''} \emph{International Journal of
Obesity} 14: 34--38.

\leavevmode\vadjust pre{\hypertarget{ref-Dplyr}{}}%
Wickham, Hadley, Romain François, Lionel Henry, and Kirill Müller. 2021.
\emph{Dplyr: A Grammar of Data Manipulation}.
\url{https://CRAN.R-project.org/package=dplyr}.

\leavevmode\vadjust pre{\hypertarget{ref-Knitr}{}}%
Xie, Yihui. 2021. \emph{Knitr: A General-Purpose Package for Dynamic
Report Generation in r}. \url{https://CRAN.R-project.org/package=knitr}.

\end{CSLReferences}



\end{document}
